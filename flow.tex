% Created 2020-12-07 seg 21:23
% Intended LaTeX compiler: pdflatex
\documentclass[11pt]{article}
\usepackage[utf8]{inputenc}
\usepackage[T1]{fontenc}
\usepackage{graphicx}
\usepackage{grffile}
\usepackage{longtable}
\usepackage{wrapfig}
\usepackage{rotating}
\usepackage[normalem]{ulem}
\usepackage{amsmath}
\usepackage{textcomp}
\usepackage{amssymb}
\usepackage{capt-of}
\usepackage{hyperref}
\author{John Doe}
\date{\today}
\title{Criando e configurando Shadow CLJS}
\hypersetup{
 pdfauthor={John Doe},
 pdftitle={Criando e configurando Shadow CLJS},
 pdfkeywords={},
 pdfsubject={},
 pdfcreator={Emacs 26.3 (Org mode 9.5)}, 
 pdflang={English}}
\begin{document}

\maketitle
\tableofcontents


\section{Conectando Shadow CLJS no terminal}
\label{sec:org474c02a}
\begin{itemize}
\item SPC m C: cider connect cljs
\item selecionar localhost com a porta 9000
\item selecionar shadow
\item selecionar a build :app
\end{itemize}


\section{Criando as pastas e documentos}
\label{sec:orgb65807e}
Primeiramente iniciamos um projeto node com o comando abaixo:

\begin{verbatim}
yarn init -y
\end{verbatim}

Com esse comando a pasta \texttt{package.json} é criada:

\begin{verbatim}
ls
\end{verbatim}

Então criamos a pasta onde ficará o source do nosso código:

\begin{verbatim}
mkdir -p src/cljs/app
\end{verbatim}

E colocamos o inicio da código:

\begin{verbatim}

(ns app.core)

(defn ^export init []
 (js/console.log "Oi mundo" ))

\end{verbatim}

Depois criamos uma pasta onde ficarão as configurações da biblioteca \texttt{shadow.cljs}:


No arquivo shadow-cljs.edn configuramos com os seguintes parâmetros:

\begin{itemize}
\item :source-paths is where you put source code
\item :dependencies is ClojureScript deps of this projects
\item :builds is where you specify build configurations
\item :app is a build id named by us, we will need it to run specific compilations from command line tools or from APIs
\item :output-dir decides where the generated files are saved
\item :asset-path is the base path of files of the hot updated code
\item :target is short for “compilation target”, :browser is for browser apps
\item :modules specifies the modules we want to generate, :main also means the bundle will be named main.js
\item :init-fn specifies the main function of the whole program
\item :devtools specifies configurations for development tools
\item :dev-http tells shadow-cljs to serve the folder target/ on port 8080.

\item :main vai ser o nome do JavaScript compilado
\item :devtools :http-root onde os arquivos da build e o arquivo html fica,
\end{itemize}


\begin{verbatim}
mkdir -p resources/public/js
\end{verbatim}


\begin{verbatim}
{:source-paths ["src/cljs"]
 :nrepl {:port 9000}
 :dependencies [[cider/cider-nrepl "0.25.5"]]
 :builds {:app {:target :browser
                :output-dir "resources/public/js"
                :modules {:main {:init-fn app.core/init}}
                :devtools {:http-root "resources/public"
                           :http-port 3000}}}
 }

\end{verbatim}

Depois disso adicionamos a biblioteca no projeto Node com o comando:

\begin{verbatim}
yarn add shadow-cljs

\end{verbatim}


Depois disso no arquivo package.json, adicionar o seguinte código, dentro das chaves, não esquecer de colocar a vírgula no fim do primeiro bloco do JSON:

\begin{verbatim}
"scripts": {"start": "shadow-cljs watch app"}
\end{verbatim}


Depois disso adicionamos um arquivo HTML para ser o host do nosso arquivo JS compilado.

\begin{verbatim}
<!DOCTYPE html>

<html lang="pt-br">
    <head>
        <meta charset="UTF-8"/>
        <meta name="viewport" content="width=device-width, initial-scale=1.0"/>
        <title >Document</title>
    </head>
    <body>
        <script src="/js/main.js">
        </script>
    </body>
</html>
\end{verbatim}


Depois podemos iniciar o programa com com o script \texttt{start} escrito acima:

\begin{verbatim}
yarn start
\end{verbatim}


\section{Instalando dependencias do react}
\label{sec:orgd9c320d}
\begin{verbatim}
yarn add react react-dom create-react-class
\end{verbatim}


\section{Instalando e configurando tailwind}
\label{sec:orgcee6bd9}
Full Stack Clojure Contact Book - [4] Front End Preparation
\url{https://www.youtube.com/watch?v=Jf94HNmCXyU}

\begin{itemize}
\item Instalando autoprefixer 9.8.6, postcss e postcss-cli para corrigir bug de compilação do tailwind
\end{itemize}

\begin{verbatim}
yarn add tailwind onchange autoprefixer postcss postcss-cli
\end{verbatim}

\begin{itemize}
\item Configurando
\end{itemize}
\begin{verbatim}
mkdir -p resources/public/css

\end{verbatim}

Precisamos criar 2 arquivos
Este primeiro na raiz do arquivo
\begin{verbatim}

module.exports = {
  purge: [],
  theme: {
    container: {
      center: true
    },
    extend: {},
  },
  variants: {},
  plugins: [],
}

\end{verbatim}

Este segundo onde o CSS vai ficar:

\begin{verbatim}
@tailwind base;

@tailwind components;

@tailwind utilities;
\end{verbatim}

Depois adicionar o seguinte script no package.json:

\begin{verbatim}
"build-styles": "tailwind build src/cljs/tailwind.css -o resources/public/css/main.css"
\end{verbatim}



\section{Lispcast: Understandig Reframe}
\label{sec:org3df8718}
\subsection{01 Reframe Stack Overview}
\label{sec:org37570a8}
\begin{itemize}
\item DOM (Document Object Model): API do browser/javascript para mudar os elementos da página, tanto o HTML e quanto os eventos. É difícil manter o estado da aplicação e o DOM sincronizados.

\item O React foi criado para lidar com os eventos de forma reativa com o DOM Virtual, e melhorou a forma como lidar com os eventos.

\item O Reagent é um wrapper para o React criado para o ClojureScript, usa o Hiccup para gerar HTML. Cria os átomos que dizem quando a view precisa ser renderizada novamente.

\item Re-Frame é um framework para aplicações. Ajuda a capturar e manter as intenções do usuário em um único local. Além disso, controla os efeitos que estão fora do DOM (requisições API, falar com bancos de dados).
\end{itemize}
\subsection{02 Getting Set Up}
\label{sec:orgc30e97d}
\begin{verbatim}
git clone https://github.com/lispcast/understanding-re-frame.git
&& cd understanding-re-frame
&& git checkout -f 001
\end{verbatim}
\end{document}
